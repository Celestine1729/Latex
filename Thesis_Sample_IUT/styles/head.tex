\documentclass[12pt]{report}
\renewcommand{\baselinestretch}{1.45}
\usepackage{amsthm,amssymb,amsmath,mathrsfs,graphicx,xcolor,float}
\usepackage{mathtools}
%%%%%%%%%%%%%%%%%%%%%%%%
\usepackage{tkz-tab,tikz,fancyhdr}
\usetikzlibrary{decorations.markings}
\tikzstyle{vertex}=[circle, draw, inner sep=0pt, minimum size=5pt]
\usetikzlibrary{arrows}
\usepackage{pgfplots}
\usetikzlibrary{snakes}
\usepackage{url}
\usepackage{setspace}
\usepackage{etoolbox}
\usepackage{caption}
\usepackage{verbatim}
\usepackage{fancyvrb}
\usepackage{tabularx}
\usepackage{moreverb} 
\usepackage{array}
% ==========+==========#==========+==========#==========+==========#==========+==========#==========+==========
% پکیج برای الگوریتم و کد
\usepackage{algorithm}
\usepackage{dsfont}
%\usepackage{arevmath}     % For math symbols
\usepackage[noend]{algpseudocode}
% ==========+==========#==========+==========#==========+==========#==========+==========#==========+==========
% جهت خلاصه‌سازی شماره‌های منابع
\usepackage[numbers,sort&compress]{natbib}
% ==========+==========#==========+==========#==========+==========#==========+==========#==========+==========
\usepackage{ifthen}
\usepackage{calc}
% ==========+==========#==========+==========#==========+==========#==========+==========#==========+==========
\usepackage[pagebackref=false,colorlinks,linkcolor=blue,citecolor=magenta]{hyperref}

% چنانچه قصد پرینت گرفتن نوشته خود را دارید، خط بالا را غیرفعال و  از دستور زیر استفاده کنید چون در صورت استفاده از دستور زیر‌‌،
% لینک‌ها به رنگ سیاه ظاهر خواهند شد و برای پرینت گرفتن، مناسب‌تر است
%\usepackage[pagebackref=false,colorlinks=black,linkcolor=black,citecolor=black]{hyperref}
% ==========+==========#==========+==========#==========+==========#==========+==========#==========+==========
\usepackage[top=3cm, bottom=3cm, left=2.5cm, right=3cm]{geometry}
% با فعال کردن دستور زیر، می‌توانید قالب دور صفحه را مشاهده کرده و از خروج نوشته‌ها جلوگیری کنید
%\usepackage[showframe=true,top=3cm, bottom=3cm, left=2.5cm, right=3cm]{geometry}
% ==========+==========#==========+==========#==========+==========#==========+==========#==========+==========
\usepackage{tocbibind}
\usepackage{multicol}
% ==========+==========#==========+==========#==========+==========#==========+==========#==========+==========
% فراخوانی بسته زی‌پرشین و دستورات مربوط به نوع فونت‌ها
\usepackage[extrafootnotefeatures]{xepersian}
\threecolumnfootnotes
\usepackage{./styles/chapterhead}
 %دستوری برای تغییر نام کلمه «کتاب‌نامه» به «منابع»
\renewcommand{\bibname}{کتاب‌نامه}
% ==========+==========#==========+==========#==========+==========#==========+==========#==========+==========
\settextfont[Scale=1.1]{Yas}
%\setromantextfont[Scale=1.1]{Times New Roman}
\setlatintextfont[Scale=1.1]{Times New Roman}
%  برای فارسی کردن اعداد در فرمول‌ها  
%\setdigitfont[Scale=1.1]{Yas}
\setdigitfont[Scale=1.1]{Parsi Digits}
\defpersianfont\percentfont[Scale=1.1]{Zar}
% فونت پارسی دیجیت علامت درصد ندارد. برای همین به‌طور دستی در قسمت commands دستوری برای آن ساخته شده است. 
% اگر لاتک شما درست باشد، تنها خطای مربوط به کد از پارسی دیجیت را دریافت خواهید کرد که مربوط به همین علامت درصد هست.
% تعریف قلم‌های فارسی و انگلیسی برای استفاده در بعضی از قسمت‌های متن
\defpersianfont\titr[Scale=1]{XB Titre}
\defpersianfont\nastaliq[Scale=.9]{IranNastaliq}
\deflatinfont\tnr[Scale=.9]{Times New Roman}
\defpersianfont\lotoos[Scale=1]{XB Roya}
\defpersianfont\elmi[Scale=1]{XB Roya} 
% ==========+==========#==========+==========#==========+==========#==========+==========#==========+==========
\def\theoremname{قضیه}
\def\definitionname{تعریف}
\def\corollaryname{نتیجه}
\def\examplename{مثال}
\def\lemmaname{لم}
\def\notename{تذکر}
\def\propname{گزاره}
\def\remname{تبصره}

% ==========+==========#==========+==========#==========+==========#==========+==========#==========+==========
\newtheorem{theorem}{\theoremname}[chapter]
\newtheorem{corollary}[theorem]{ \corollaryname}
\newtheorem{definition}[theorem]{\definitionname }
\newtheorem{example}[theorem]{\examplename   }
\newtheorem{lemma}[theorem]{\lemmaname   }
\newtheorem{note}[theorem]{\notename   }
\newtheorem{proposition}[theorem]{\propname   }
\newtheorem{remark}[theorem]{\remname   }

% ==========+==========#==========+==========#==========+==========#==========+==========#==========+==========
% در این بخش می‌توان نحوه ترتیب شماره‌گذاری‌ها را تغییر داد.

%\renewcommand{\thetheorem}{\arabic{theorem}.\arabic{section}.\arabic{chapter}}
%\renewcommand{\thefigure}{\arabic{figure}-\arabic{chapter}}
%\renewcommand{\theequation}{\arabic{equation}}

%\renewcommand{\thetheorem}{\arabic{theorem}-\arabic{chapter}}
\renewcommand{\thetheorem}{\arabic{chapter}-\arabic{theorem}}
%\renewcommand{\thefigure}{\arabic{figure}-\arabic{chapter}}
\renewcommand{\thefigure}{\arabic{chapter}.\arabic{figure}}
\renewcommand{\theequation}{\beginL\arabic{equation}\endL}

% ==========+==========#==========+==========#==========+==========#==========+==========#==========+==========
%%%% انتهای اثبات قضایا و لمها
\newcommand{\eqd}{\ \  \blacksquare\hfill}

\setlength{\unitlength}{1cm}
% ==========+==========#==========+==========#==========+==========#==========+==========#==========+==========
%%%%%%%%%%%%%%  تعریف دستوری که سر صفحه‌ها را در ابتدای  قسمتهایی که فصل نیستند، مثل ضمیمه، واژه‌نامه، ... 
\newcommand{\headerb}[1]{\markright{ #1 \ \  \hrulefill\ \  }}
% ==========+==========#==========+==========#==========+==========#==========+==========#==========+==========
% تعریف دستوری برای قسمت فهرست کلمات اختصاری
%\newcommand{\symbb}[3]{ \makebox[14cm] {\makebox[2cm][c]{\hfill}\makebox[2.5cm][r]{$\displaystyle #1$} \ \  \makebox[7.5cm][r]{#2} \ \ \makebox[3cm][l]{#3}}\newline}
\newcommand{\symbb}[3]{ \makebox[15cm] {\makebox[1cm][c]{\hfill}\makebox[2cm][r]{$\displaystyle #1$} \ \  \makebox[10cm][r]{#2} \ \ \makebox[2cm][l]{#3}}\newline}
% ==========+==========#==========+==========#==========+==========#==========+==========#==========+==========
% تعریف دستوری برای قسمت فهرست نمادها
\newcommand{\symb}[3]{ \makebox[12.5cm] {\makebox[2cm][c]{\hfill}\makebox[4.1cm][r]{$\displaystyle #1$} \ \  \makebox[6.4cm][r]{#2} \ \ \makebox[2cm][l]{#3}}\newline}
% ==========+==========#==========+==========#==========+==========#==========+==========#==========+==========
% دستوری برای تعریف واژه‌نامه انگلیسی به فارسی
\newcommand{\englishTOfarsi}[2]{#2 \dotfill \lr{#1}\newline} 
% دستوری برای تعریف واژه‌نامه فارسی به انگلیسی 
\newcommand{\farsiTOenglish}[2]{#1 \dotfill \lr{#2}\newline}
% ==========+==========#==========+==========#==========+==========#==========+==========#==========+==========
\makeatletter
\renewcommand\chapter{\if@openright\cleardoublepage\else\clearpage\fi
%                    \thispagestyle{empty}% original style: plain
                    \global\@topnum\z@
                    \@afterindentfalse
                    \secdef\@chapter\@schapter}
\makeatother



