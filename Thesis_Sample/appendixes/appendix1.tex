\chapter*{پیوست‌ 1}
\thispagestyle{empty}\fancyhead[LE]{\nouppercase\thepage\ \hfill\ پیوست ۱}
\fancyhead[LO]{\nouppercase پیوست ۱ \ \hfill\ \thepage}
\addcontentsline{toc}{chapter}{پیوست‌ 1}
در این بخش الگوریتم‌های مورد نیاز در فصل ۳ رساله که به وسیله‌ی نرم‌ افزار GAP نوشته شده‌اند ارائه شده است. 
خروجی الگوریتم  Prog-I علاوه بر آن زیرمجموعه‌های سه عضوای از $\mathbb{Z}_n $ که  گراف کیلی جمعی حاصل همبند و صحیح است، 
قدرمطلق مقادیر ویژه‌ی گراف مذکور  نیز است. 
 
\hspace{1cm}
\begin{latin}
\noindent \bf Prog-I:
\end{latin}
$------------------------------------------$
\begin{code}
Main:=function(n)
local i,j,k,r,t,a,c,w,A,B,G,T,D;
G:=CyclicGroup(IsPermGroup,n);
a:=MinimalGeneratingSet(CyclicGroup(IsPermGroup,n))[1];
T:=Elements(List(G,x->x^2));     B:=Set([]);
for i in [1..n-1] do
  for j in [i+1..n-1] do
    for k in [j+1..n-1] do
      w:=0; A:=[]; D:=[];
      for r in [0..n-1] do
        c:=E(n)^(i*r)+E(n)^(j*r)+E(n)^(k*r);
        t:=ImaginaryPart(c)^2+RealPart(c)^2;
        if IsInt(t)  and IsInt(Sqrt(t)) then   w:=w+1;
        fi;
      od;
      if w=n and Intersection([a^i,a^j,a^k],T)=[]
         and Group(a^i,a^j,a^k)=G
         and Size(Group(a^(i-j),a^(i-k),a^(j-k)))>=Size(G)/2
      then
         Add(D,i); Add(D,j); Add(D,k); Add(A,D);
         for r in [0..n-1] do
           c:=E(n)^(i*r)+E(n)^(j*r)+E(n)^(k*r);
           t:=ImaginaryPart(c)^2+RealPart(c)^2;
           if IsInt(t) then  t:=Sqrt(t);
           fi;
           Add(A,t);
         od;
         Add(B,A);
      fi;
    od;
  od;
od;
return B;
end;
\end{code}